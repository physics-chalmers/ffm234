%%
%% Automatically generated file from DocOnce source
%% (https://github.com/hplgit/doconce/)
%%
%%
% #ifdef PTEX2TEX_EXPLANATION
%%
%% The file follows the ptex2tex extended LaTeX format, see
%% ptex2tex: http://code.google.com/p/ptex2tex/
%%
%% Run
%%      ptex2tex myfile
%% or
%%      doconce ptex2tex myfile
%%
%% to turn myfile.p.tex into an ordinary LaTeX file myfile.tex.
%% (The ptex2tex program: http://code.google.com/p/ptex2tex)
%% Many preprocess options can be added to ptex2tex or doconce ptex2tex
%%
%%      ptex2tex -DMINTED myfile
%%      doconce ptex2tex myfile envir=minted
%%
%% ptex2tex will typeset code environments according to a global or local
%% .ptex2tex.cfg configure file. doconce ptex2tex will typeset code
%% according to options on the command line (just type doconce ptex2tex to
%% see examples). If doconce ptex2tex has envir=minted, it enables the
%% minted style without needing -DMINTED.
% #endif

% #define PREAMBLE

% #ifdef PREAMBLE
%-------------------- begin preamble ----------------------

\documentclass[%
oneside,                 % oneside: electronic viewing, twoside: printing
final,                   % draft: marks overfull hboxes, figures with paths
10pt]{article}

\listfiles               %  print all files needed to compile this document

\usepackage{relsize,makeidx,color,setspace,amsmath,amsfonts,amssymb}
\usepackage[table]{xcolor}
\usepackage{bm,ltablex,microtype}

\usepackage[pdftex]{graphicx}

\usepackage[T1]{fontenc}
%\usepackage[latin1]{inputenc}
\usepackage{ucs}
\usepackage[utf8x]{inputenc}

\usepackage{lmodern}         % Latin Modern fonts derived from Computer Modern

% Hyperlinks in PDF:
\definecolor{linkcolor}{rgb}{0,0,0.4}
\usepackage{hyperref}
\hypersetup{
    breaklinks=true,
    colorlinks=true,
    linkcolor=linkcolor,
    urlcolor=linkcolor,
    citecolor=black,
    filecolor=black,
    %filecolor=blue,
    pdfmenubar=true,
    pdftoolbar=true,
    bookmarksdepth=3   % Uncomment (and tweak) for PDF bookmarks with more levels than the TOC
    }
%\hyperbaseurl{}   % hyperlinks are relative to this root

\setcounter{tocdepth}{2}  % levels in table of contents

% prevent orhpans and widows
\clubpenalty = 10000
\widowpenalty = 10000

\newenvironment{doconceexercise}{}{}
\newcounter{doconceexercisecounter}


% ------ header in subexercises ------
%\newcommand{\subex}[1]{\paragraph{#1}}
%\newcommand{\subex}[1]{\par\vspace{1.7mm}\noindent{\bf #1}\ \ }
\makeatletter
% 1.5ex is the spacing above the header, 0.5em the spacing after subex title
\newcommand\subex{\@startsection{paragraph}{4}{\z@}%
                  {1.5ex\@plus1ex \@minus.2ex}%
                  {-0.5em}%
                  {\normalfont\normalsize\bfseries}}
\makeatother


% --- end of standard preamble for documents ---


% insert custom LaTeX commands...

\raggedbottom
\makeindex
\usepackage[totoc]{idxlayout}   % for index in the toc
\usepackage[nottoc]{tocbibind}  % for references/bibliography in the toc

%-------------------- end preamble ----------------------

\begin{document}

% matching end for #ifdef PREAMBLE
% #endif

\newcommand{\exercisesection}[1]{\subsection*{#1}}

\input{newcommands_keep}

% ------------------- main content ----------------------



% ----------------- title -------------------------

\thispagestyle{empty}

\begin{center}
{\LARGE\bf
\begin{spacing}{1.25}
FFM234, Klassisk fysik och vektorfält - Veckans tal
\end{spacing}
}
\end{center}

% ----------------- author(s) -------------------------

\begin{center}
{\bf \href{{http://fy.chalmers.se/subatom/tsp/}}{Christian Forssén}, Institutionen för  fysik, Chalmers${}^{}$} \\ [0mm]
\end{center}

\begin{center}
% List of all institutions:
\end{center}
    
% ----------------- end author(s) -------------------------

% --- begin date ---
\begin{center}
Aug 10, 2019
\end{center}
% --- end date ---

\vspace{1cm}


% --- begin exercise ---
\begin{doconceexercise}
\refstepcounter{doconceexercisecounter}

\subsection{Uppgift 2.4.8}

Betrakta vektorfältet
$$
\vec E(\vec{r}) = \frac{m}{4\pi r^3} (2\cos \theta \hat{r} + \sin\theta \hat{\theta}),
$$
där $m$ är en konstant. (Detta är fältet från en elektrisk dipol.)

Bestäm ekvationen för den fältlinje till $\vec E(\vec{r})$ som går genom punkten $(r, \theta, \varphi) = (2, \pi /4, \pi /6)$.

% --- begin hint in exercise ---

\paragraph{Hint.}
Fältlinjer är de kurvor som följer ett vektorfält på så sätt att de i varje punkt har vektorfältet som sin tangentvektor. Fältlinjer kan parametriseras $\vec{r} = \vec{r}(\tau)$ och differentialekvationerna för att bestämma dem är
$$
\frac{\mbox{d}\vec{r}}{\mbox{d}\tau} = C \vec{E},
$$
där $C$ är en godtycklig konstant vilken ju inte påverkar tangentriktningen.

Med cartesiska koordinater gäller ju att förskjutningsvektorn $\mbox{d}\vec{r}$ kan skrivas $\mbox{d}\vec{r} = \hat{x} \mbox{d}x + \hat{y} \mbox{d}y + \hat{z} \mbox{d}z$ och vektorekvationen ovan ger tre differentialekvationer (en för varje riktning $\hat{x}$, $\hat{y}$, $\hat{z}$):
$$
\left\{
\begin{array}{ll}
x: &
\frac{\mbox{d}x}{\mbox{d}\tau} = C E_x \\
y: &
\frac{\mbox{d}y}{\mbox{d}\tau} = C E_y \\
z: &
\frac{\mbox{d}z}{\mbox{d}\tau} = C E_z.
\end{array}
\right.
$$
Men om fältet är mycket enklare att uttrycka i kroklinjiga koordinater är det fördelaktigt att teckna differentialekvationerna i dessa riktningarna istället. Men då får man komma ihåg att förskjutningsvektorn blir
$$
\mbox{d}\vec{r} = \sum_{i=1}^3 h_i \hat{u}_i \mbox{d}u_i,
$$
där $h_i$ är koordinatsystemets skalfaktorer.

% --- end hint in exercise ---


% --- begin answer of exercise ---
\paragraph{Answer.}
$r = 4 \sin^2 \theta$, $\varphi = \pi/6$.

% --- end answer of exercise ---


% --- begin solution of exercise ---
\paragraph{Solution.}
Att göra

% --- end solution of exercise ---

% Closing remarks for this Exercise

\paragraph{Remarks.}
Uppgiften illustrerar hur man ställer upp differentialekvationerna för fältlinjer i fallet då vektorfältet enklast beskrivs i ett kroklinjigt koordinatsystem. Den illustrerar också hur en specifik fältlinje kan identifieras från ett randvillkor.


\end{doconceexercise}
% --- end exercise ---


% ------------------- end of main content ---------------

% #ifdef PREAMBLE
\end{document}
% #endif

