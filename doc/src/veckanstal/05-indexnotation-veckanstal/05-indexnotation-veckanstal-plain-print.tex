%%
%% Automatically generated file from DocOnce source
%% (https://github.com/hplgit/doconce/)
%%
%%


%-------------------- begin preamble ----------------------

\documentclass[%
oneside,                 % oneside: electronic viewing, twoside: printing
final,                   % draft: marks overfull hboxes, figures with paths
10pt]{article}

\listfiles               %  print all files needed to compile this document

\usepackage{relsize,makeidx,color,setspace,amsmath,amsfonts,amssymb}
\usepackage[table]{xcolor}
\usepackage{bm,ltablex,microtype}

\usepackage[pdftex]{graphicx}

\usepackage[T1]{fontenc}
%\usepackage[latin1]{inputenc}
\usepackage{ucs}
\usepackage[utf8x]{inputenc}

\usepackage{lmodern}         % Latin Modern fonts derived from Computer Modern

% Hyperlinks in PDF:
\definecolor{linkcolor}{rgb}{0,0,0.4}
\usepackage{hyperref}
\hypersetup{
    breaklinks=true,
    colorlinks=true,
    linkcolor=linkcolor,
    urlcolor=linkcolor,
    citecolor=black,
    filecolor=black,
    %filecolor=blue,
    pdfmenubar=true,
    pdftoolbar=true,
    bookmarksdepth=3   % Uncomment (and tweak) for PDF bookmarks with more levels than the TOC
    }
%\hyperbaseurl{}   % hyperlinks are relative to this root

\setcounter{tocdepth}{2}  % levels in table of contents

% prevent orhpans and widows
\clubpenalty = 10000
\widowpenalty = 10000

\newenvironment{doconceexercise}{}{}
\newcounter{doconceexercisecounter}


% ------ header in subexercises ------
%\newcommand{\subex}[1]{\paragraph{#1}}
%\newcommand{\subex}[1]{\par\vspace{1.7mm}\noindent{\bf #1}\ \ }
\makeatletter
% 1.5ex is the spacing above the header, 0.5em the spacing after subex title
\newcommand\subex{\@startsection*{paragraph}{4}{\z@}%
                  {1.5ex\@plus1ex \@minus.2ex}%
                  {-0.5em}%
                  {\normalfont\normalsize\bfseries}}
\makeatother


% --- end of standard preamble for documents ---


% insert custom LaTeX commands...

\raggedbottom
\makeindex
\usepackage[totoc]{idxlayout}   % for index in the toc
\usepackage[nottoc]{tocbibind}  % for references/bibliography in the toc

%-------------------- end preamble ----------------------

\begin{document}

% matching end for #ifdef PREAMBLE

\newcommand{\exercisesection}[1]{\subsection*{#1}}

\input{newcommands_keep}

% ------------------- main content ----------------------



% ----------------- title -------------------------

\thispagestyle{empty}

\begin{center}
{\LARGE\bf
\begin{spacing}{1.25}
FFM234, Klassisk fysik och vektorfält - Veckans tal
\end{spacing}
}
\end{center}

% ----------------- author(s) -------------------------

\begin{center}
{\bf \href{{http://fy.chalmers.se/subatom/tsp/}}{Christian Forssén}, Institutionen för  fysik, Chalmers${}^{}$} \\ [0mm]
\end{center}

\begin{center}
% List of all institutions:
\end{center}
    
% ----------------- end author(s) -------------------------

% --- begin date ---
\begin{center}
Aug 10, 2019
\end{center}
% --- end date ---

\vspace{1cm}


Här följer ledtrådar till två roliga, men kluriga, uppgifter från kapitel 5: Indexnotation.



% --- begin exercise ---
\begin{doconceexercise}
\refstepcounter{doconceexercisecounter}

\subsection*{5.5.10}

Bevisa den Stokes-analoga satsen 
\begin{equation*}
\oint\limits_{\partial S}d\vec{r}\times\vec v
       =\int_S(d\vec S\times\vec{\nabla})\times\vec v.
\end{equation*}
Visa att ett val
av ytan $S$ i $xy$-planet reproducerar Greens formel.

% --- begin hint in exercise ---

\paragraph{Hint.}
\begin{itemize}
\item Bilda ett vektorfält $\vec{F} = \vec{a} \times \vec{v}$, där $\vec{a}$ är ett godtyckligt, men konstant vektorfält. 

\item Teckna Stokes sats för detta nya vektorfält $\vec{F}$. Målet är sedan att skriva både VL och HL av Stokes sats som $\vec{a}$ gånger en integral. För att nå dit får man skriva om några kryssprodukter för vilket man med fördel kan använda indexformalism.

\item Målet är alltså att komma fram till att
\end{itemize}

\noindent
$$
-\vec{a} \cdot \oint\limits_{\partial S}d\vec{r}\times\vec v = -\vec{a} \cdot \int_S(d\vec S\times\vec{\nabla})\times\vec v.
$$
\begin{itemize}
\item Eftersom det sambandet skall gälla för godtyckligt fält $\vec{a}$ så måste integralerna vara lika.
\end{itemize}

\noindent
% --- end hint in exercise ---

\end{doconceexercise}
% --- end exercise ---




% --- begin exercise ---
\begin{doconceexercise}
\refstepcounter{doconceexercisecounter}

\subsection*{5.5.11}

Visa att arean av en plan yta omsluten av en kurva $C$ är
\begin{equation*}
A=\frac{1}{2} \left| \,\oint\limits_C\vec{r}\times d\vec{r}\, \right|.
\end{equation*}

% --- begin hint in exercise ---

\paragraph{Hint.}
\begin{itemize}
\item Lägg ett koordinatsystem så att den plana ytan ligger i $xy$-planet.

\item Notera sedan att $\vec{r}\times d\vec{r}$ är en vektor som pekar i $z$-riktningen. Eftersom vi skall ha absolutvärdet av integralen kan vi få fram vektorns enda komponent genom att skalärmultiplicera med $\hat{z}$.

\item Notera att $\vec{z} \cdot (\vec{r}\times d\vec{r}) = d\vec{r} \cdot (\hat{z} \times \vec{r})$.

\item Då kommer man till en punkt där man kan utnyttja Stokes sats. 

\item Sedan får man utnyttja indexformalism för att skriva om den dubbla kryssprodukten. Notera att vektorn $\hat{z}$ kan skrivas som $\delta_{3l}$ med indexformalism (detta skall ju tolkas som tre komponenter, $l=1,2,3$, där bara den tredje är nollskild och lika med 1).
\end{itemize}

\noindent
% --- end hint in exercise ---

\end{doconceexercise}
% --- end exercise ---


% ------------------- end of main content ---------------

\end{document}

