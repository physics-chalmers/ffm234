%%
%% Automatically generated file from DocOnce source
%% (https://github.com/hplgit/doconce/)
%%
%%
% #ifdef PTEX2TEX_EXPLANATION
%%
%% The file follows the ptex2tex extended LaTeX format, see
%% ptex2tex: http://code.google.com/p/ptex2tex/
%%
%% Run
%%      ptex2tex myfile
%% or
%%      doconce ptex2tex myfile
%%
%% to turn myfile.p.tex into an ordinary LaTeX file myfile.tex.
%% (The ptex2tex program: http://code.google.com/p/ptex2tex)
%% Many preprocess options can be added to ptex2tex or doconce ptex2tex
%%
%%      ptex2tex -DMINTED myfile
%%      doconce ptex2tex myfile envir=minted
%%
%% ptex2tex will typeset code environments according to a global or local
%% .ptex2tex.cfg configure file. doconce ptex2tex will typeset code
%% according to options on the command line (just type doconce ptex2tex to
%% see examples). If doconce ptex2tex has envir=minted, it enables the
%% minted style without needing -DMINTED.
% #endif

% #define PREAMBLE

% #ifdef PREAMBLE
%-------------------- begin preamble ----------------------

\documentclass[%
oneside,                 % oneside: electronic viewing, twoside: printing
final,                   % draft: marks overfull hboxes, figures with paths
10pt]{article}

\listfiles               %  print all files needed to compile this document

\usepackage{relsize,makeidx,color,setspace,amsmath,amsfonts,amssymb}
\usepackage[table]{xcolor}
\usepackage{bm,ltablex,microtype}

\usepackage[pdftex]{graphicx}

\usepackage[T1]{fontenc}
%\usepackage[latin1]{inputenc}
\usepackage{ucs}
\usepackage[utf8x]{inputenc}

\usepackage{lmodern}         % Latin Modern fonts derived from Computer Modern

% Hyperlinks in PDF:
\definecolor{linkcolor}{rgb}{0,0,0.4}
\usepackage{hyperref}
\hypersetup{
    breaklinks=true,
    colorlinks=true,
    linkcolor=linkcolor,
    urlcolor=linkcolor,
    citecolor=black,
    filecolor=black,
    %filecolor=blue,
    pdfmenubar=true,
    pdftoolbar=true,
    bookmarksdepth=3   % Uncomment (and tweak) for PDF bookmarks with more levels than the TOC
    }
%\hyperbaseurl{}   % hyperlinks are relative to this root

\setcounter{tocdepth}{2}  % levels in table of contents

% prevent orhpans and widows
\clubpenalty = 10000
\widowpenalty = 10000

\newenvironment{doconceexercise}{}{}
\newcounter{doconceexercisecounter}


% ------ header in subexercises ------
%\newcommand{\subex}[1]{\paragraph{#1}}
%\newcommand{\subex}[1]{\par\vspace{1.7mm}\noindent{\bf #1}\ \ }
\makeatletter
% 1.5ex is the spacing above the header, 0.5em the spacing after subex title
\newcommand\subex{\@startsection{paragraph}{4}{\z@}%
                  {1.5ex\@plus1ex \@minus.2ex}%
                  {-0.5em}%
                  {\normalfont\normalsize\bfseries}}
\makeatother


% --- end of standard preamble for documents ---


% insert custom LaTeX commands...

\raggedbottom
\makeindex
\usepackage[totoc]{idxlayout}   % for index in the toc
\usepackage[nottoc]{tocbibind}  % for references/bibliography in the toc

%-------------------- end preamble ----------------------

\begin{document}

% matching end for #ifdef PREAMBLE
% #endif

\newcommand{\exercisesection}[1]{\subsection*{#1}}

\input{newcommands_keep}

% ------------------- main content ----------------------



% ----------------- title -------------------------

\thispagestyle{empty}

\begin{center}
{\LARGE\bf
\begin{spacing}{1.25}
FFM234, Klassisk fysik och vektorfält - Veckans tal
\end{spacing}
}
\end{center}

% ----------------- author(s) -------------------------

\begin{center}
{\bf Christin Rhen, Chalmers${}^{}$} \\ [0mm]
\end{center}

\begin{center}
% List of all institutions:
\end{center}
    
% ----------------- end author(s) -------------------------

% --- begin date ---
\begin{center}
Aug 10, 2019
\end{center}
% --- end date ---

\vspace{1cm}


% --- begin exercise ---
\begin{doconceexercise}
\refstepcounter{doconceexercisecounter}

\subsection{Uppgift 9.4.10: Vektorfält från Greensfunktion}

På en sfär med radien $a$ befinner sig en ytkälla med tätheten $\sigma(\vec r)=\sigma_0\cos\theta$, där $\theta$ är vinkeln från en punkt på sfären. Bestäm vektorfältet för en punkt på symmetriaxeln.

% --- begin hint in exercise ---

\paragraph{Hint.}
\begin{itemize}
\item Betrakta punkter $\vec{r} = z\hat{z}$, dvs längs symmetriaxeln.

\item Teckna Greensfunktionen $G(\vec{r},\vec{r}\,') = (4\pi|\vec{r}-\vec{r}\,'|)^{-1}$ och fundera på hur avståndet kan uttryckas när $\vec{r}=z\hat{z}$.

\item Ytkällan motsvarar en laddningsfördelning $\rho(\vec{r}) = \sigma_0 \cos\theta \delta(r-a)$.

\item Resultatet blir en ganska krånglig integral där variablsubstitutionen $t = z^2 +a^2−2za\cos\theta'$ kan komma väl till pass.

\item Notera sedan att man får olika fall beroende på tecknen på $(z+a)$ och $(z-a)$.
\end{itemize}

\noindent
% --- end hint in exercise ---


% --- begin answer of exercise ---
\paragraph{Answer.}
\begin{equation}
\vec F = \left\{
\begin{array}{ll}
-\frac{\sigma_0}3\hat z, & |z| < a \\
\frac{2\sigma_0a^3}{3|z|^3}\hat z, & |z|>a
\end{array}
\right.
\end{equation}

% --- end answer of exercise ---


% --- begin solution of exercise ---
\paragraph{Solution.}
Eftersom att punkten $\vec r$ befinner sig på ytkällans symmetriaxel, kan vi definiera ett kartesiskt koordinatsystem så att denna symmetriaxel sammanfaller med $z$-axeln. Då blir $\theta$ samma vinkel som den vi normalt kallar $\theta$ i sfäriska koordinater, och $\vec r=z\hat z$.

Vektorfältet $\vec F(\vec r)=-\nabla\phi(\vec r)$, där potentialen $\phi(\vec r)$ ges av
\begin{equation}
    \phi(\vec r)=\int_{\mathrm R^3}G(\vec r,\vec r')\rho(\vec r')\mathrm dV'.
\end{equation}
Eftersom att vi är i tre dimensioner är Greensfunktionen $G(\vec r,\vec r')=[4\pi|\vec r-\vec r'|]^{-1}$. Den angivna laddningstätheten är en ytkälla, så $\rho(\vec r\,')=\delta(r'-a)\sigma_0\cos\theta'$. Vi betraktar $\vec{r} = z\hat{z}$ och får alltså 
\begin{align}
    \phi(z\hat z)&=\int_{\mathbb R^3}\frac{\delta(r'-a)\sigma_0\cos\theta'}{4\pi\sqrt{z^2+r'^2-2zr'\cos\theta'}}r'^2\sin\theta'\mathrm dr'\mathrm d\theta'\mathrm d\varphi'\\
    &=\frac{\sigma_0a^2}2\int_0^\pi \frac{\cos\theta'\sin\theta'}{\sqrt{z^2+a^2-2za\cos\theta'}}\mathrm d\theta'.
\end{align}
Vi gör nu variabelsubstitutionen 
\begin{gather}
t=z^2+a^2-2za\cos\theta',\nonumber\\
\mathrm dt=2za\sin\theta'\mathrm d\theta',\nonumber\\
\theta'=0\to t=(z-a)^2,\nonumber\\
\theta'=\pi\to t=(z+a)^2.
\end{gather}
Nämnaren i integranden blir nu $\sqrt t$, och $\sin\theta'$ i täljaren blir en del av $\mathrm dt$. Det kvarvarande $\cos\theta'$ i täljaren får vi genom att lösa ut $\cos\theta'$ ur definitionen av $t$. 

Integralen är nu
\begin{align}
    \phi(\vec r)&=\frac{\sigma_0}{4z^2}\int_{(z-a)^2}^{(z+a)^2} \frac{z^2+a^2-t}{2\sqrt t}\mathrm dt\\
    &=\frac{\sigma_0}{4z^2}\left[(z^2+a^2)\int_{(z-a)^2}^{(z+a)^2} \frac1{2\sqrt t}\mathrm dt-\int_{(z-a)^2}^{(z+a)^2} \frac{\sqrt t}2\mathrm dt\right]\\
    &=\frac{\sigma_0}{4z^2}\left[(z^2+a^2)\Big[\sqrt t\Big]_{(z-a)^2}^{(z+a)^2} -\frac13\Big[t^{3/2}\Big]_{(z-a)^2}^{(z+a)^2}\right]\\
    &=\frac{\sigma_0}{4z^2}\left[(z^2+a^2)\Big[|z+a|-|z-a|\Big]-\frac13\Big[|z+a|^3-|z-a|^3)\Big]\right].
\label{phi}
\end{align}
Kom ihåg att för alla reella tal $x$ så är $\sqrt{x^2}=|x|$. Vi får alltså fyra fall -- $|z|>a$ och $|z|<a$ kombinerat med $z>0$ och $z<0$ -- som måste behandlas var för sig. 

\paragraph{Inuti sfären.}
Eftersom att $a>z$ så är $|z+a|=z+a$ och $|z-a|=a-z$, oavsett om $z$ är positiv eller negativ. Två av de fyra fallen kan alltså studeras samtidigt. 

Insättning av absolutbeloppen ovan i ekvation~(\ref{phi}) ger, efter en smula algebra, att $\phi(\vec r)=\sigma_0 z/3$. Vektorfältet blir då
\begin{equation}
    \vec F =-\hat z\frac{\partial\phi}{\partial z}=-\frac{\sigma_0}3\hat z.
\end{equation}

\paragraph{Utanför sfären.}
Här måste positiva och negativa $z$ studeras var för sig.

När $z>a>0$ är $|z+a|=z+a$ och $|r-a|=r-a$. Substitution i~(\ref{phi}) och förenkling av resultatet ger att $\phi(\vec r)=\sigma_0a^3/3z^2$. Vektorfältet blir 
\begin{equation}
    \vec F =-\hat z\frac{\partial\phi}{\partial r}=\frac{2\sigma_0a^3}{3z^3}\hat z,\quad z>0. \label{z+}
\end{equation}

När $|z|>a$ men $z<0$ så blir $|z+a|=-(z+a)$ och $|z-a|=-(z-a)$. Den här gången blir $\phi(\vec r)=-\sigma_0a^3/3z^2$ och 
\begin{equation}
    \vec F =-\hat z\frac{\partial\phi}{\partial r}=-\frac{2\sigma_0a^3}{3z^3}\hat z,\quad z<0.\label{z-}
\end{equation}

Notera att alla kvantiteter utom $z$ i~(\ref{z+}) och~(\ref{z-}) är konstanter; de ändrar alltså aldrig tecken. Därför kan vi kombinera dessa två ekvationer, och uttrycka fältet överallt utanför sfären som
\begin{equation}
    \vec F =-\hat z\frac{\partial\phi}{\partial r}=-\frac{2\sigma_0a^3}{3|z|^3}\hat z.
\end{equation}

\paragraph{Kontroll: ytladdning.}
Notera att vårt fält, betraktat längs symmetriaxeln, har en diskontinuitet vid $z=\pm a$. Denna motsvarar ytladdningen. T.ex., vid $\theta = 0$ (dvs $z=+a$) har vi enligt uppgiftsformuleringen en ytladdning med styrkan $\sigma_0$. Detta kan vi verifiera genom att räkna ut
\begin{equation}
\hat{z} \cdot \left( \vec{F}_+ - \vec{F}_- \right)_{z=a} = \frac{2\sigma_0}{3} - \frac{-\sigma_0}{3} = \sigma_0.
\end{equation}

% --- end solution of exercise ---

\end{doconceexercise}
% --- end exercise ---


% ------------------- end of main content ---------------

% #ifdef PREAMBLE
\end{document}
% #endif

